\chapter*{Abstract}
The term \emph{technical debt} has been used to described the increased cost of changing or maintaining a system due to expedient shortcuts taken during development, possibly due to budget or time constraints. The term has gained significant attention in the agile and academic community and several scientific models have been proposed to keep track of and solve technical debt.\\\\
Tribler, a platform to share and discover content in a complete decentralized way, has accumulated a tremendous amount of technical debt over the last ten years of scientific research in the area of peer-to-peer networking.
The platform suffers from a complex architecture, unintuitive user interface, an incomplete, unstable testing framework and a significant amount of unmaintained code.
A new simple, flexible and component-based architecture that readies Tribler for the next decade of research is proposed and discussed.
We lay the foundations for this new architecture by implementing a flexible, convenient RESTful API and a new graphical user interface.\\\\
Additional work includes paying off various kind of technical debt by the means of a major refactoring in the testing framework, several heavy modifications within the core of Tribler and improvements to the infrastructure to make it more usable and robust.
With the deletion of 12.581 lines, the modification of 765 lines and addition of 12.429 lines, we show that we contributed to the increase of several important software metrics and paid off a huge amount of technical debt. Raising awareness about the accumulated debt is of uttermost importance if we wish to prevent the deterioration of the system.\\\\
Our experiments will demonstrate that the performance of Tribler has not significantly degraded due to the invasive modifications as performed in this thesis work. We perform some static analysis to verify the usability of various components in the system and propose future work for the components that require more attention.