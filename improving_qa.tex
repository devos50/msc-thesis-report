\label{chapter:refactoring}
\chapter{Improving Software Metrics and Refactoring}
In Chapter \ref{chapter:problem-description}, the huge accumulated amount of technical debt in the system has been highlighted. A new robust and future-proof architecture has been proposed in Section \ref{chapter:architecture}. In this Chapter, the most important activities during the span of this thesis work will be presented. This includes a developer-friendly REST API and, a modern, simplistic and user-friendly grahical user interface and major important modifications to the available testing framework.

\section{General Overview and Statistics}
According to GitHub, 155 Pull Requests have been accepted, which account for a total of 401 commits. In total, 27.659 lines have been added and 12.581 lines have been removed. 765 files have been modified. These statistics excludes the removal of the wx GUI code which account for 12.429 deletions.

\section{Improving tests and testability}
The most fundamental way to verify the correctness of software detect issues as soon as possible in the development cycles, is by having an exhaustive test suite. As described in Chapter \ref{chapter:problem-description}, the current test suite is plagued with instable and non-functional tests. This section will discuss the performed work to strengthen and stabilize the test suite.

\subsection{Improving Code Coverage}
Code coverage is defined as the percentage of source code that is covered by at least one test in the test suite. Our continuous integration environment offers tools to track the code coverage over time. After each automated test suite execution a comprehensive report with detailed information about the code coverage is generated\todo{plaatje?}.\\\\
To make sure that the responsibility of code coverage is not neglected in future work on Tribler, an addition check for each pull request has been added that verifies that the code contributed in the respective pull request is covered by tests. While not created by the author of this thesis, this check is an effective way to keep the code coverage metric under control.

\subsection{External Network Resources}
On of the problems with the test suite was that dependencies on external network resources should either be removed or one should verify that the resources are under the control of the developer and always available. The test suite contains various tests where external torrent files are fetched from the internet, in particular, from the Ubuntu repository. While this repository can guarantee high availability, any downtime in this external resource can lead to failing tests. The implemented solution for this design flaw is to start up a local HTTP server that serves the torrent file. While this approach requires more code to manage this local server, it completely removes the dependency on the Ubuntu repository.\\\\
The same solution has been applied to solve the dependency on external seeders. A small number of tests makes assumptions on the availability of torrent pieces of the network. This certainly makes tests fail if the executing machines has a bad or even no internet connection. The process of setting up a local seeder session is straightforward. Again, this approach requires code to properly start and shut down the seeder session. The implementation is reusable to an extend that developers of tests can reuse the implemented solutions with only a few lines of code.\\\\
Unfortunately, there are some external network dependencies left which are considered harder to refactor. A handful of tests are performing a remote keyword search, requiring various communities in Dispersy to be loaded. These tests are dependent on available peers in the respective community in order to make sure there are incoming search results. The proposed solution here is to start various dedicated Dispersy sessions on the local machine. Due to time constraints, the implementation of this solution is considered future work.

\subsection{Instability of Tests}
Well-designed tests should only fail if some new code is breaking existing functionality. If no changes are presents, the tests should always succeed. Reducing dependencies on external network is not sufficient to guarantee this in Tribler. The structural problem of the tests is that the system is infested with race conditions. Race conditions can be invisible since they often occur in a very specific runtime setting of the system, making the debugging process of these kind of errors frustrating.\\\\
During this thesis, many race conditions have been detected and solved. One interesting observation is that some issues only occurred on a specific platform. We believe can be explained by differences in the implementation of underlying threading model across operating systems. The most common cause of failing tests can be addressed to delayed calls in the Twisted reactor. During the test execution, Tribler is restarted many times. If a developer leaves by accident a delayed call behind when the shut down procedure has been completed, this delayed call might be executed in the wrong Tribler session, possibly leading to an inconsistent state of the system. Making sure the reactor is completely clean is not straightforward: if one is not aware of scheduled calls in the system, the mistake is easily made.\\\\
Writing stable tests also requires the test to be limited in what they do. Each test should only be focussed on the specific part of the system that has to be tested\todo{cite?}. While often unnecessary, a significant amount of the available tests are focused on starting a complete Tribler session, testing a small subset of the system, and shutting down Tribler again. While this approach is relatively easy to code, starting a fully-fledged session often leads to more instability and unexpected side-effects during test execution. Instead, only the classes to be tested should be instantiated and any dependencies this class have, should be mocked. Mocking ensures that developers have control over dependencies, allowing them to specify any expected return value. Moreover, the execution time of these small unit tests is significantly lower than the tests where a Tribler session is managed. The additional unit tests that have been written during this thesis, are following the described design.

\subsection{Multi-Platform}
During the past 10 years of contributions to Tribler, several platform-specific workarounds have been implemented. Since tests are only executed on one platform (Linux), code that runs only in a Windows or OS X environment is never tested. This problem can be solved by running the tests on multiple platforms. This will allow developers to detect defects on other platforms more earlier in the development process. By aggregating the generated coverage report on each platform, the code coverage metric should also increase.\\\\
The setup of the testing environments on Windows and OS X is straightforward. New slave nodes to specify the Windows and OS X test runners have been created. The tests on OS X are executed on a Mac Mini\todo{specs}. In order to run the tests on Windows, two virtual machines using Proxmox have been created, both 32-bit and 64-bit environments. In total, the tests are executed on four platforms now: Linux, Windows 32-bit, Windows 64-bit and OS X. So far, the OS X and Windows test executers have completed over 2.500 test runs. Each test runner generates a coverage reports and these reports are merged in the final analyse step in the build pipeline.

\section{REST API}
As described in Chapter \ref{chapter:architecture}, communication between the graphical user interface and the Tribler core is facilitated using a REST API. This Section explains the implementation of the API in more detail.\\\\
The REST API is written using Twisted. While there are plenty of Python libraries available that allow developers to facilitating a web server in their application, Twisted is used since it already represents a big part of the internal structure of Tribler. With the possibility to integrate the REST API into the main application flow, we avoid having to create special constructions to run the API for instance on a separate thread. API endpoints in Twisted are represented as a resource tree. This is in accordance with REST where the URL of the request can be treated like a path in the resource tree.\\\\
Except for some endpoints, all data returned by the API is structured in JSON format. The JSON format is well adopted and easy to parse. Most of the endpoints are straightforward implementations where the client performs a request and some data is returned. There are situations where the client does a request and a stream of data should be returned. For instance, this is the case when the user performs a search query. Sometimes, data should be returned to the client, even if the client did not ask for this data. When a crash in the Tribler core code occurred, the client should be notified of this crash and possibly warn the user that he or she should restart the application. For this purpose, an asynchronous events stream has been designed and created. Clients can open this event stream and interesting notifications are sent over this stream. All messages that are sent over the \emph{events} connection are shown in Table x\todo{maken}.\\\\
The API is started in two stages. Just before starting Tribler, the \emph{events} connection is activated. This connection is initialized earlier so we can send interesting updates of the upgrader to the client. When the upgrader is completed and Tribler has been started, the other endpoints are activated and the API is ready to be used.

\section{Graphical User Interface}


\section{Improving Sofware Artifacts}
