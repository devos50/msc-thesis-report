\chapter{Problem Description}

The goal of this thesis project is to help Tribler mature from an experimental research prototype into production-level code with potentially reliable usage by millions of users.\\\\
After careful analysis it was decided that within the context of a nine month project the strongest contribution to the future of Tribler would be a step forward in quality assurance. At this point we believe the project does not need a particular focus on feature improvements, novel additional features, or boosting performance. After over ten years of software development by 44 unique contributors the investment in quality assurance has been neglected.\\\\
Tribler suffers from a lack of unit tests, includes race conditions, lacks documentation, non-functional regression test framework, and has significant technical debt. The small amounts of tests that are available, are highly unstable, leading to wasted development time.\\\\
This thesis is focussed on a round of invasive maintenance and cleaning of the code and all other infrastructure such as installers and testing environment. Our work aims to ensure that it is possible to conduct another decade of experimental distributed systems research with the Tribler code base. The alternative is continued usage and expansion of the code, which are likely to lead to a forced clean slate approach.\\\\
The structural problem is the lack of maintenance capacity. Each contributor to the Tribler research in the form of a bachelor, master, or PhD student needs to be primarily focussed on their thesis work. A thesis requires concrete experimental results, contribution to theory, or both. We believe the lack of student enthusiasm for fixing bugs and writing documentation is the root cause of current state of the code base.\\\\
In this remaining of this chapter, various problems within the Tribler project will be highlighted. These issues can roughly be divided in two categories: issues related to the neglect of quality assurance and issues related to the code base itself.

	
\section{Quality Assurance}
The ongoing lack of any quality assurance measure has led to several concerns. In this Section, some of the problems regarding quality assurance will be addressed.

\subsection{Test Suite}
Tribler has a structural lack of proper designed unit tests. Currently, there are 99 tests and 48.970 lines of code in the Tribler module (excluding code in the Dispersy framework). Many of these tests are taking over half a minute to complete and are bootstrapping an extensive Tribler session. Only a small fraction of the test suite has the characteristics of unit tests. Having tests that are doing a broad range of operations, inevitably leads to undesired side-effect and failing tests. No single attempt has been made to mock components of the system to simplify tests and focus on the part of the system that has to be verified.\\\\
To illustrate this problematic situation in more detail, the tunnel community is taken as example. The tunnel community allows users of Tribler to anonymously download content and is one of the most anticipated features of the software. The current test suite contains no single unit test that is focussed on this part of the code. Instead, there are several unstable integration-like tests that are starting up a graphical user interface and perform an anonymous download. While one might argue that having such a test might be sufficient for regression testing purposes, this test does not fully cover the source code and any code related to error handling is completely uncovered.\\\\
There is one more factor that contributes to the instability of the current test suite. A significant part of the test suite is depending on external network resources, ranging from trackers and seeders for a specific torrent to other peers in the decentralized Dispersy network. This fragile architecture gives rise to failing tests due to unavailable nodes, unexpected responses from external peers and other unpredicted circumstances.\\\\
In general, well designed tests exclude any dependency on external resource that is outside the control of the developer. This can be achieved by mocking method calls to return dummy data. Additionally, one can make sure that the external resource is available in the local testing environment. For instance, when a test is dependent on a specific torrent seeder, a local libtorrent session can be started that seeds this torrent.

\subsection{Testing Environment}
Tribler uses a Jenkins environment to automate the testing of new Pull Requests (PRs). These tests are only executed on a Linux server. Tribler itself, however, is also packaged and distributed for Windows and OS X. According to the GitHub download statistics, the Windows and OS X distributions account for 91\% of the downloads\footnote{https://api.github.com/repos/tribler/tribler/releases/latest}. Not running any test on this platform is a missed opportunity to identify platform-specific errors. Moreover, any conditional in the source code that is only executed on a specific platform, remains completely uncovered and untested.

\subsection{Documentation}
The responsibility to maintain a proper and up-to-date documentation base for current and new developers has been completely neglected during the lifespan of the Tribler project. Except for some general information about the project, there is a minimal amount of information available about the system. Since the process of familiarization with the source code is very hard, Tribler has become an unattractive open-source project to contribute to.

\subsection{Outdated Dependencies}
The Tribler code base has many dependencies on other libraries. At the time of writing, Tribler depends on at least 20 other python libraries. Apart from that, we need libraries for testing and packaging the code. Keeping these libraries up to date, is a necessary and important process.\\\\
Outdated libraries might lead to compatibility issues and workarounds. For packaging Tribler on Windows, we are using \emph{py2exe}, a library that is not maintained since 2008. Usage of \emph{py2exe} might be dropped in favor of more maintained and mature libraries, for instance, \emph{PyInstaller}.


\section{Code quality}
In the previous Section, the lack of quality assurance and the consequences of this has been elaborated. In this Section, the focus will shift to the code itself. Identified issues in the current structure of the source code will be discussed.

\subsection{Graphical User Interface}
The current Graphical User Interface (GUI) of Tribler is written with the wxPython library. The GUI accounts for just over 20.000 source lines of code (SLOC), making up 33\% of the Tribler code base which is significant.\\\\
The code base of the GUI is has been subject to various refactoring cycles. This leads to a very complex and hard to understand package of code. Technical debt is clearly visible in this part of the code base. Moreover, the interface is plagued with many useless and hidden features that are contributing to the complexity of the code. A clean slate approach with a more mature and user-friendly libraries seems to be the solution that could benefit Tribler the most.\\\\
The current decade of software engineering provides many alternative tools and libraries that allows to build a visually appealing, platform-independent interface with a minimal amount of code. For instance, the popular and mature Qt framework allows developers to specify their layout in a visual designer.

\subsection{High Coupling between modules}
The user interface and core of Tribler are interleaved to such an extent that it is inconvenient for developers to test out specific features without making changes to the GUI. Providing a minimal Tribler service that only runs the core code and removing the dependencies between the core and user interface will boost the productivity of developers. An interface can be provided to provide necessary  interactions with the core. There are some more occurrences of dependencies between modules which should be removed. For instance, a full Tribler session object is passed to the constructor of the main database class while a path to the destination location of the database file suffices.\\\\
By separating modules from each other, the ease of maintainability can be improved since developers do not need to worry about errors that are propagating to other module dependencies. Separation also leads to a code base that is easier to understand.

\subsection{Inconsistent Code Style}
After the contribution of 44 developers, numerous code styles across the code base can be identified. This leads to an inconsistent code base. There is absolutely no appliance to the PEP8 guidelines which is based on the fact that the current source code contains a sheer amount of 11644 violations.

%Refactoring overview 5-pages

%This chapter show a general overview of all the various large and small contributions made to boost the health of the code base. In total over 13000 lines of code have been altered. 33 pull requests accepted.
%GUI screenshots, before, after
%remove DB calls from GUI.
%Github screenshots with altered lines of code.
%progressive JSON API.

%Quality Assurance infrastructure 8-pages

%In the previous chapter a general overview is provided of the numerous changes to the code base. Now we focus on the specific changes for quality assurance in more detail. Key change is 
%splitting of the code base into two parts to make them easier to maintain.  
%-documentation, easier to contribute, improved learning curve, REST API.
%software performance regresssion-graphs
%evolution of code coverage

%Experimental results 10-pages GRAPHS

%We now focus on performance experiments of our new Tribler code base and demonstrate the key improvements.

%Startup time graph, 10 runs + 10 runs new
% + breakdown, checkpoints..
%remote search
%download speed
%memory usage+stress test, while doing lots of REST API calls.
%content discovery speed, Subscribe channels experiment: 1,5,10,15,25 joined channels and resource usage.

%...