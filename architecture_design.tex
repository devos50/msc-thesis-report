\chapter{Architecture and Design}
\label{chapter:architecture}

In this chapter, we first focus on the evolution of the Tribler architecture throughout the last decade of research. Better understanding of the architectural and design decisions that have been taken in the past, will help us to shed light on the question what contributed to the current state of the Tribler system. Next, we will propose a new future-proof architecture that can be adopted in the next generation of Tribler.\\\\
According to the Open Hub tool\cite{openhubtribler} which gathers statistics about many open-source projects available on the Internet, Tribler received code contributions from 111 unique contributors so far. This list is most likely not exhaustive since some work of contributors might have been finished by other members of the Tribler team or has never been merged into the main code base. A search query for \emph{Tribler} in the repository of Delft University of Technology\footnote{http://repository.tudelft.nl}, yields a total of 66 search results, consisting of 35 results which are contributions in the form of a MSc or BSc thesis and 31 research-oriented papers in the form of a PhD dissertation or (published) research work.\\\\
The remainder of this chapter will present a historical view of the evolution of the Tribler platform, starting in 2007 and concluding with the proposal of a new, robust and scalable architecture that is ready for the next decade of research.

\section{From ABC to Tribler: A Social-based Peer-to-Peer System}
In April 2005, Tribler started out as a fork of the Another BitTorrent Client (ABC) application, an improved BitTorrent client. ABC is based on BitTornado which extended from the BitTorrent core system, originally written by Bram Cohen. ABC was shipped with a user interface and a variety of features to manage BitTorrent downloads. The software utilizes the BitTorrent engine, at that time completely written in Python like the rest of ABC. This is most probably the reason why the Python programming language is used for Tribler development. In March 2006, the first public version of Tribler got released with version code 3.3.4.\\\\
In 2007, the first major research paper was published, describing Tribler as a social-based peer-to-peer system\cite{pouwelse2008tribler}. The key idea as described in this paper is that social connections between peers in a decentralized network can be exploited to increase usability and performance of the network. This is based on the idea that peers belonging to a social group are not likely to steal (free-ride) bandwidth from each other. Free-riding is a phenomena that is widely observed in the regular BitTorrent network. The system architecture of Tribler as described in the work of Pouwelse et al. is presented in Figure \ref{fig:tribler-architecture-2008}. We will now highlight the most important components of this architecture.

\begin{figure}[t]
	\centering
	\includegraphics[width=1.0\columnwidth]{images/tribler_architecture_2007}
	\caption{The system architecture of Tribler as described in \cite{pouwelse2008tribler}.}
	\label{fig:tribler-architecture-2008}
\end{figure}

\subsection{Collaborative Downloads}
The BitTorrent engine provides tools to download and seed files in a decentralized way using a BitTorrent-compatible protocol. In addition, the module allows usage of the collaborative downloader feature which significantly increases download speed by exploiting idle upload capacity of online friends in the network. The implemented protocol to facilitate these collaborative downloads is called 2Fast and it uses social groups where members who trust each other, collaborate to improve their download performance.\\\\
The 2Fast protocol works as follows: peers that are participating in a social group are either \emph{collectors} or \emph{helpers}. A collector is a peer that is interested in obtaining a complete copy of a particular file whereas a helper is a peer that is recruited by a collector to help downloading that file. Both types of peers start downloading a file using the regular BitTorrent protocol and the collaborative download extensions. However, before a helper tries to obtain a file piece in the network, it first asks the collector for approval which is granted when no other helpers have downloaded or are downloading the file piece in question already. Afterwards, the helper peer sends the piece to the collector. According to the performed experiments, the maximum achievable speed-up is 4 and 8 times respectively For ADSL and ADSL-2 internet connections.

\subsection{Geo-Location Engine}
On the left side of the architecture in Figure \ref{fig:tribler-architecture-2008}, we notice the \emph{Geo-Location Engine}, \emph{Peer Geo-Location Engine} and the \emph{map} component in the user interface. The \emph{Geo-Location Engine} is used to determine the physical location of other peers in the torrent swarm, using the open hostip API\footnote{http://hostip.info}. The \emph{Peer Geo-Location Engine} has been built on top of this module, providing the primitives to display the location of peers on a map in the user interface. This feature stems from the goal to ease the process of visual identification of potential collaborators.

\subsection{Content Discovery and Recommendation}
On the right side of the architecture in Figure \ref{fig:tribler-architecture-2008}, we identify components to facilitate content discovery and recommendations. The \emph{BuddyCast} algorithm is designed to serve recommendations to users and to enable peer and content discovery. BuddyCast is an epidemic protocol which works as follows: each peer in the network maintains a number of taste buddies with their content preference lists and a number of random peers, void of any information about their content preferences. Periodically, BuddyCast performs an \emph{exploration} or \emph{exploitation} step: when an exploration step is executed, the peer connects to one of its taste buddies. When an exploitation step is performed, the peer connects to a random peer in the network. When the connection with the other peer is successful, a BuddyCast message is exchanged, containing the identities of a number of known taste buddies along with their top-10 preference lists, a number of random peers, and the top-50 content preferences of the peer. The age of each peer is included in the message to help other users know about the "freshness" of peers. After the BuddyCast messages are exchanged, the received information is stored in the local database of each peer, called the \emph{Preference Cache}. Information about discovered peers are stored in the \emph{Peer Cache}. To limit the exchange of redundant information, each peer maintains a list of recently contacted peers.\\\\
The BuddyCast mechanism interacts with the user interface in two different ways. On the \emph{Files I Like} page in the interface, each peer indicates its preference for certain files expressed as a score between 1 and 5. Initially, this list is filled with the most recent downloads of the peers. Second, the user interfaces displays similar taste buddies and facilitates a content browser where each item is annotated with an estimated interest indicator for that user.

\section{Tribler Between 2007 and 2012}
The first version in the next generation of Tribler, version 4.0, was released in 2007\cite{tribler4tf}. Many features from the 3.x release cycles are untouched and some new functionalities have been added, most notable in the user interface. With a new embedded video player, users are able to play videos (while being downloaded) directly from within the user interface. This video player is powered by the popular \emph{VLC} library\footnote{http://www.videolan.org/vlc/} and bindings that facilitates video management and playback in various popular user interface libraries. Tribler 4.0 allowed users to search for content inside the Tribler network but also supported searches of content available on YouTube\footnote{https://www.youtube.com} and LiveLeak\footnote{http://www.liveleak.com}. These search results were presented in a YouTube-like thumbnail grid to the user. The user interface of Tribler 4.0 is displayed in Figure \ref{fig:tribler4}.\\

\begin{figure}[h!]
	\centering
	\includegraphics[width=1.0\columnwidth]{images/tribler4}
	\caption{The user interface of Tribler 4.0.}
	\label{fig:tribler4}
\end{figure}

\noindent The development of Tribler continued with the release of version 5.0 in 2009\cite{historyoftribler}. The user interface has been subject to a complete redesign, introducing a dark theme, which was replaced by a white theme shortly after release. The focus of Tribler 5.0 has been on the stability and performance of remote content search and the download mechanism. The thumbnails have been dropped in favour of a paginated list.\\\\
Tribler 5.1 contained some major improvements to the user interface, thanks to the feedback of the community. A new, novel addition in Tribler 5.2 is the concepts of channels, similar to YouTube. One goal of the organization of content into channels was to prevent spam inside the network by favouring content available in more popular channels. Whereas custom widgets with an own look-and-feel have been used in this version, they all got replaced in Tribler 5.3 by native buttons to create a more natural feel on each supported platform. Additionally, a tag cloud with popular keywords have been added to the home page of Tribler to help users determine which content they possibly want to look for. The paginated list was replaced by a single, scrollable list of items. In the next release, Tribler 5.4, a \emph{magic search} feature has been implemented where similar search results are collapsed using text similarity functions and digit extraction. The usefulness of this feature is apparent when searching for content that is split into several parts such as a sequel of books or a television show. This idea is that this feature leads to a much cleaner and comprehensive results list when searching for content.\\\\
The final release in the 5.x series, Tribler 5.9, bought some major additions. The complete BuddyCast core has been rewritten, moving away from a TCP overlay to an implementation based on UDP, providing benefits to the compatibility with \emph{Network Address Translation (NAT)} firewalls. Tribler adopted the \emph{Peer-to-peer streaming peer protocol (PPSPP)} protocol\cite{bakker2015peer}, implemented in \emph{libswift}\cite{libswiftgithub}, PPSPP provides download capabilities over UDP, thus removing the TCP layer from the BitTorrent engine.\\\\
The architecture around the time of Tribler version 5.5 is depicted in Figure \ref{fig:tribler-core-architecture-55}. We notice that this architecture is significantly more complex compared to the design as presented in Figure \ref{fig:tribler-architecture-2008}. The architecture excludes the structure of the user interface which is equally complex. This model is the result of gradually adding and modifying smaller components in Tribler that have been developed during research, such as 2fast, BuddyCast and the \emph{Secure Overlay}, providing a high-level communication mechanism. We notice four different threads that need to work together, contributing to the complexity of the code since developers need to be aware of context switches (the jumping between different threads during execution of the application). Whereas the project contained around 45.000 lines of code in 2007, this number has increased to over 90.000 in 2010, doubling in size.

\begin{figure}[h!]
	\centering
	\includegraphics[width=1.0\columnwidth]{images/architecture/core_architecture_55}
	\caption{An overview of the core architecture around the time of Tribler 5.5.}
	\label{fig:tribler-core-architecture-55}
\end{figure}

\section{Tribler between 2012 and 2016}
Shortly after the release of Tribler 5.9, version 6.0 is released where a new user interface has been implemented and the PPSPP protocol implementation has been replaced by the libtorrent library, written in C++\footnote{https://github.com/arvidn/libtorrent}. This release contained minor bug fixes that increased performance and usability in general. After the release of version 6.0, several smaller releases (6.1, 6.2 and 6.3) followed. The focus of the Tribler platform shifted toward the realisation of anonymous downloads and end-to-end encryption, designed and implemented by Plak, Tanaskoski and Ruigrok in 2014 and 2015\cite{plak2014anonymous}\cite{tanaskoski2014anonymous}\cite{ruigrok2015bittorrent}. The Tribler 6.4 release shipped an experimental anonymous download mechanism and hidden seeding services implementation. Additionally, the release also introduced a Trivial File Transfer Protocol (TFTP)\cite{sollins1992tftp} implementation, a simplified version of the popular File Transfer Protocol (FTP)\cite{postel1985rfc}, commonly used to transfer files over the Internet. TFTP in Tribler is utilized to exchange torrent files between peers in Tribler when searching for content. The Tribler 6.4.1 release contained some major security fixes after an external code review by a member on the Tor mailing list\cite{githubissue1066}.\\\\
We will now describe two major dependencies of Tribler: Dispersy and Twisted. These dependencies are crucial for a correct functioning of Tribler and a basic knowledge of the working of these dependencies is beneficial when contributing to Tribler.

\subsection{Dispersy}
With the release of Tribler 6.1, \emph{Dispersy} got introduced as dependency. Described in \cite{zeilemaker2013dispersy} and mainly developed by Zeilemaker and Schoon, Dispersy lies at the foundations of the messaging and synchronization system in Tribler and is designed to deliver messages reliably in unpredictable networks. It provides a NAT traversal mechanism to improve the extent to which users are connectable in the network. Dispersy provides tools to create distinct overlay networks, called \emph{communities}, that peers can join and where messages can be exchanged. The implemented communities in Tribler, together with a short description, is presented in Table \ref{table:dispersy-communities}. While being a major dependency of Tribler, an extensive analysis of Dispersy is considered outside the scope of this thesis.

\begin{table}
	\begin{tabularx}{\textwidth}{|l|X|}
		\hline
		\textbf{Community name} & \textbf{Purpose} \\ \hline
		\emph{AllChannel} & Used to discover new channels and to perform remote channel search operations.\\ \hline
		\emph{BarterCast4} & While currently disabled, this community was used to spread statistics about the upload and download rates of peers inside the network and has originally been created as a mechanism to prevent free-riding in Tribler\cite{meulpolder2009bartercast}.\\ \hline
		\emph{Channel} & This community represents a single channel and is responsible for managing torrents and playlists inside that channel.\\ \hline
		\emph{Multichain} & The Multichain community utilizes blockchain technology and can be regarded as the accounting mechanism that keeps track of shared and used bandwidth.\\ \hline
		\emph{Search} & This community contains functionalities to perform remote keyword searches for torrents and torrent collecting operations.\\ \hline
		\emph{(Hidden)TunnelCommunity} & The (hidden) tunnel community contains the implementation of the Tor-like protocol that enables anonymity when downloading content and contains the foundations of the hidden seeder services protocol, used for anonymous seeding.\\ \hline
	\end{tabularx}
	\caption{An overview and description of implemented Dispersy communities in Tribler as of July 2016.}
	\label{table:dispersy-communities}
\end{table}

\subsection{Twisted}
\label{subsec:architecture-twisted}
In 2014, it was decided to make significant changes to the architecture by utilizing the Twisted library, an event-driven networking engine written in Python\footnote{https://twistedmatrix.com/trac/}. Twisted allows programmers to write code in an asynchronous way. The utilization of Twisted has been motivated by the presence of various callback mechanisms in Tribler as can be identified in Figure \ref{fig:tribler-core-architecture-55}. The library provides a simple model for handling callbacks and events. At the heart of Twisted, we find the \emph{reactor} which is the implementation of the event loop\cite{twistedreactoroverview}. The event loop is a programming construct that waits for and dispatches events or messages in a Python application.\\\\
The threading model as present in Tribler 6 has been illustrated in Figure \ref{fig:old-threading-model}. In most applications that are using Twisted, the reactor operates on the main thread of a Python application. In Tribler, the reactor runs on a separate thread since the main thread is occupied by the event loop of wxPython, the library used to realise the user interface. This means that code performing operations with the user interface such as a refresh of a list, should always be executed on the main thread in Python. Operations that are using Twisted constructs however, should be scheduled on the reactor thread to function correctly. The Twisted threadpool provides a pool of additional threads to dispatch work to and can be utilized for longer-running operations that should not block the main or reactor thread. To make context switching more easy to implement, several method decorators have been implemented by Tribler developers, visible in Figure \ref{fig:old-threading-model}.\\\\
Developers should always be aware of the threading context when implementing new features or modifying existing ones. Long blocking calls on the main thread should be avoided as much as possible since they lead to an unresponsive user interface during the execution of that method. However, database calls should be scheduled on the reactor thread. While this architecture reduced the number of threads we have to manage compared to Figure \ref{fig:tribler-core-architecture-55}, we are still stuck with a dedicated thread for the reactor and complex thread switching patterns.

\begin{figure}[h!]
	\centering
	\includegraphics[width=1.0\columnwidth]{images/architecture/threading_model_tribler}
	\caption{The threading model used by Tribler 6, together with the primitives to schedule operations on different threads.}
	\label{fig:old-threading-model}
\end{figure}

\section{The Roadmap of Tribler}
\label{sec:tribler-roadmap}
In the previous section, the evolution of Tribler has been discussed. We have illustrated the increased complexity in terms of the architecture, design and threading model over time. We now turn our attention to the future of Tribler and propose a new architecture where we address some of the design flaws introduced in previous development iterations of Tribler. This new architecture should prepare Tribler for another ten year of research. We start by defining three requirements that our new architecture should meet:
\begin{itemize}
	\item \emph{simplicity}: we wish to shift towards an architecture that has a better learning curve for new developers that are not familiar with the code base. The current architecture is hard to learn, prone to errors and has a complex threading model, increasing the time for new developers to get familiar with the code. By simplifying the architecture, we save developer time and increase possibilities for contributions from external developers outside the Tribler organization.\\\\
	Creating a simplified architecture inevitably leads to decisions whether to remove unnecessary, unused or broken components. The current architecture has various features for which we believe that the maintenance costs outweighs the benefits of that particular feature for end users. Making decisions to remove some components and thus code, can lead to a significant reduction of technical debt.
	\item \emph{flexibility}: by introducing a sufficient level of flexibility in the system, developers can focus on the development of individual components when contributing. While we can identify many different modules in Figure \ref{fig:tribler-core-architecture-55}, there are still numerous interdependencies between them, making it hard to modify and test parts of the system in isolation. We propose a methodology based on component-based software engineering, where we provide interfaces between components to communicate with each other. The user interface should be implemented as a separate component, in comparison to the current architecture where the core and user interface cannot be used as separate modules (this will be elaborated in more detail in Section \ref{subsec:gui-core-packages}).
	\item \emph{focus on performance}: Tribler contains various time-critical components that should work reliably and fast. Examples of such components include the download engine, the anonymous overlay and the content discovery mechanisms. Several development tasks have been conducted that are focused on performance optimization of individual components or on the system as whole.\\\\
	The new architecture should be designed with future performance engineering in mind. An unclear and unstructured architecture causes overhead for developers when boosting performance as is for instance illustrated by the thread switching mechanisms necessaries to implement a specific feature that utilizes both the Tribler core and the user interface. By considering performance engineering as early in the process, we can adjust our architecture to allow for performance modifications in the future.
\end{itemize}
We propose the architecture depicted in Figure \ref{fig:tribler7} which design follows a layered, component-based approach. The remainder of this section will discuss components in the architecture in more detail and highlight decisions that have been made during the design process.

\begin{figure}[h!]
	\centering
	\includegraphics[width=0.9\columnwidth]{images/architecture/tribler7}
	\caption{The proposed architecture of Tribler 7, consisting of a trusted overlay (1), a content-discovery mechanism (2), libtribler (3) and a user interface (4).}
	\label{fig:tribler7}
\end{figure}

\subsection{Trusted Overlay}
The trusted overlay is the lowest layer in Tribler and provides primitives for discovering and picking new trusted peers in the network. At the lowest level of the trusted overlay, we identify the trusted walker, the central component for discovering other peers.\\\\
Currently, the Dispersy framework is responsible for discovering new peers within the network, using a gossiping protocol\cite{zeilemaker2013dispersy}. This discovery mechanism is illustrated in Figure \ref{fig:dispersy-discover} and executed at fixed time intervals. It works as follows: suppose node \emph{A} wants to discover an additional peer. First, he sends an \emph{introduction-request} message to a random peer he knows, say node \emph{B}. Node \emph{B} now replies with an \emph{introduction-reply} message, containing information about a node that \emph{B} knows, in this case node \emph{C}. Meanwhile, node \emph{B} sends a \emph{puncture-request} message to node \emph{C} which in turn punctures the NAT firewall of node \emph{A}, making sure that node \emph{A} can connect to him. This algorithm both provides a NAT-puncturing mechanism and allows a node to discover new peers.\\

\begin{figure}[h!]
	\centering
	\includegraphics[width=0.7\columnwidth]{images/architecture/dispersy_discover}
	\caption{The peer discovery and NAT puncture mechanism as implemented in Dispersy.}
	\label{fig:dispersy-discover}
\end{figure}

The described mechanism of discovering new peers using Dispersy will be replaced by a trusted walker that uses accumulated reputation in the Multichain community, the accounting mechanism that keeps track of shared and used bandwidth, providing more reputation when a user provides upload capacity to help other users. The key idea is that sybil nodes, forged identities in the network, and free riders (peers that are only interested in downloading content while not actively uploading to help other peers) are ignored and not considered as trusted peers since their reputation is low and are not likely to be selected by the trusted walker.\\\\
New peers in the network that have not accumulated any reputation yet, start out by creating some random interactions with other nodes while learning about the network and the amount of reputation of other peers. With an interval, every node runs an algorithm to calculate the reputation of their known peers. The amount of uploaded and downloaded data does not have to be the only factor of this reputation mechanism: the uptime of the user in question can also be considered, where a higher uptime might lead to a better reputation and thus a higher trustworthiness. This leads to a trust network where each node knows about other trusted peers with which they can exchange content.\\\\
Interaction with the trust overlay can be realised by using an higher-level Application Programming Interface (API) which provides the facilities to perform operations regarding the discover of new trusted peers and management of known ones. By providing a trusted overlay API, the component allows for easy reuse in other projects which is beneficial when the module will be published as an open-source project.

\subsection{Spam-Resilient Content Discovery}
Discovering content is a key feature of Tribler. The spam-resilient content discovery component allows users to discover and search for torrents, channels and playlists using peers discovered by the trusted overlay. The current implemented mechanism for information exchange in Tribler where messages are disseminated within segregated communities works well enough for this purpose, except for the exchange of more exotic  meta data such as content thumbnails. Providing users with a visual preview of content in the form of thumbnails is an opportunity to make the user interface more appealing, however, a robust implementation in a complete decentralized network is challenging due to the fact that the content can be present in a huge volume, thus increasing the total size of the thumbnails that have to be synchronized. We wish to keep overhead introduced by thumbnail synchronization to a minimum and we must have a decent filtering algorithm to avoid inappropriate imagery from being shown in the user interface. These features will be considered future work and are not discussed in the remainder of this thesis.

\subsection{libtribler}
libtribler provides primitives to developers to make use of the above described components and contains the implementation of a RESTful API that is used to communicate with it. We will now discuss the components which together account for this layer.

\subsubsection{\textbf{Download Engine}}
The download engine is one of the most crucial parts in Tribler: before facilitated by the BitTorrent and libswift libraries, the most recent version utilizes the open-source libtorrent library to facilitate decentralized downloads. libtorrent is written in the C++ programming language, however compatibility layers for various other programming languages such as Python, Go and Java are available. \emph{libtorrent} uses an alert mechanism to notify the application that is using the library about events, such as download state transitions, peer discovery in the torrent swarm or completion of a meta info lookup in the \emph{Distributed Hash Table (DHT)}. There are no benefits for replacing the current download engine with another library that allows to download torrents. Moreover, the current way libtorrent is used in Tribler requires minimal changes to adhere to the proposed architecture, except for some optional refactoring of the current code. This includes a revision of the code to remove calls to deprecated methods in the libtorrent library.\\\\
We should note that the method to fetch peers from the DHT in libtorrent is private and not accessible from Tribler. Under normal circumstances, this method is only invoked by libtorrent, however, we manually call this method when performing a DHT lookup on behalf of another peer during the execution of the hidden services protocol, described in more detail in \cite{ruigrok2015bittorrent}. To still be able to perform a lookup of peers in the DHT, we make use of a third-party library, named pymdht\footnote{https://github.com/rauljim/pymdht}, an implementation of the DHT protocol, written in Python and available on GitHub. This dependency is undesirable since it introduces extra complexity and load of the system. Effort should be made to make this desired method accessible in libtorrent so Tribler can get rid of the dependency.

\subsubsection{\textbf{Video Streaming Server}}
\label{subsubsec:video-server}
The video streaming server streams the video data to a video player outside libtribler after or during a download. The implemented video server in the current architecture listens for and serves HTTP requests and is implemented using the \emph{SimpleHTTPServer} library\footnote{https://docs.python.org/2/library/simplehttpserver.html}, a built-in Python module that can be used to build a HTTP web server. The functioning of the video server is based on HTTP range requests (where a specific range of a file is specified in the request header), allowing the web server to serve a subset of a file when the user jumps to a random time position during video playback.\\\\
The inner working of the video player is illustrated in Figure \ref{fig:video-server} and works as follows: when the user starts playing a video in an external video player, the player performs a HTTP range request to the video server implemented in Tribler (1). This range request contains information in the header about the requested byte range of the video file. When the video server receives the request, it first checks whether the requested range has been downloaded by the download engine (libtorrent) already. If so, the server returns the requested data to the client (5). If the requested range is not available, the video server notifies libtorrent that the bytes in the requested range should be prioritized for download (2), reducing the latency before the requested range is completely available. The download engine waits until all bytes are downloaded and when the requested range is downloaded, the video server is notified about this event (4) and  completes the request by sending the data to the client (5).\\

\begin{figure}[h!]
	\centering
	\includegraphics[width=0.7\columnwidth]{images/architecture/video_server}
	\caption{The sequence of operations when performing a HTTP range request when streaming a video using Tribler.}
	\label{fig:video-server}
\end{figure}

\noindent While the video server in the current form is functional, there is a performance improvement that we should consider: the video player currently runs on a separate thread. By integrating the video server inside the Twisted reactor thread, we can reduce the complexity of the server and utilize all of the functionalities that Twisted provides, for instance, managing incoming HTTP requests. An additional consideration could be to run the video server in a dedicated process, separate from Tribler. This might increase the complexity since a communication mechanism between the Tribler and video server process is required to inform libtorrent about the prioritization of pieces. Since it is easier to integrate the video server in Twisted, we might consider to create a separate process only if we encounter performance issues when running the video server.

\subsubsection{\textbf{Family Filter}}
The freedom to upload any type of content in the network, comes with a price. The legal aspect of the available content in Tribler can be disputable. While Tribler  contains legal content such as pornographic material, that content might be undesirable for most casual users. A mechanism called the family filter is  implemented to filter out content that is not always safe for display. This filter is enabled by default and uses a list of keywords that can be associated with pornographic content. Discovered torrents gets classified by this filter, based on the torrent name, file names and other meta data. Unfortunately, this ad-hoc approach is not very effective since there are quite a few false positive classifications. While it provides some basic filtering, we noticed that the keyword-based approach can be greatly improved by using a more sophisticated classification approach such as collaborative voting. However, we would consider this as an enhancement rather than a defect that prevents a correct usage of Tribler.

\subsubsection{\textbf{Credit Mining}}
Ongoing work on a credit mining system (CMS) in decentralized systems has been extensively described by the work of Capot\k{a} et al\cite{capotka2015decentralized} and is defined as the activity performed by peers for the purpose of earning credit. A possible purpose of the earned credits is to access exclusive content or receiving preferential download treatment in case of network congestion. Although the credit mining component is not enabled for end-users by default, a CMS has been implemented in Tribler, responsible for contributing bandwidth to the community without any intervention of the user. This mechanism is displayed in Figure \ref{fig:credit-mining} and works as follows: first, the user selects a source of swarms for the CMS to take in consideration. Possible swarm sources are Tribler channels, RSS feeds or a directory containing one or more torrent files. Next, the CMS periodically selects a subset of the chosen swarms by the user. Finally, Tribler joins the swarms and tries to maximize earned credits by downloading as little as possible and by maximizing the amount of uploaded data.\\

\begin{figure}[h!]
	\centering
	\includegraphics[width=0.7\columnwidth]{images/architecture/credit_mining}
	\caption{The credit mining system described in the work of Capot\k{a} et al.\cite{capotka2015decentralized}}
	\label{fig:credit-mining}
\end{figure}

\noindent The CMS can apply different policies for swarm selection. The first policy is to selects a swarm with the lowest ratio of seeders to all peers (leechers and seeders). Intuitively, this boosts swarms that are under-supplied (having a low amount of seeders). The second policy is to select swarms based on the swarm age. The intuition behind this policy is that newer content is often better seeded so it might be more beneficial to boost older swarms. The final policy that can be used is a random policy that selects a swarm using an uniform distribution.\\\\
This credit mining mechanism is a convenient way for users to increase their reputation by supplying bandwidth to the community, requiring minimal intervention. The CMS functions in conjunction with the Multichain community which is used as accounting tool to keep track of downloaded and uploaded bytes.

\subsubsection{\textbf{Channel Management}}
Tribler allows users to create their own channel and share content within that channel. Content can be shared in the form of torrents and playlists where a playlist is composed of a bundle of potentially related torrent, for instance, some episodes of a tv show. Users can add content to the channels of other users, providing that the owner of the channel has set the type of the channel to open. Other channel types include semi-open and closed where the difference between these two types is that in a closed channel, users cannot write comments on torrents.

\subsection{Communication Between the GUI and libtribler}
\label{subsec:communication-gui-libtribler}
The communication between the the GUI and libtribler should be facilitated by a \emph{Representational State Transfer (REST)} API. The REST architecture was introduced and defined by Roy Fielding in 2000\cite{fielding2000fielding} and is used frequently when building APIs that operate on the World Wide Web. A service that conforms to the REST architecture, is called RESTful.\\\\
In a RESTful architecture, resources and collections, identified by Uniform Resource Identifiers (URIs), are returned by servers to clients that performed a request. Common HTTP operations (\emph{GET}, \emph{POST}, \emph{PUT} and \emph{DELETE}) are used to manipulate or retrieve these resources and collections. Table \ref{table:rest-api-operations} provides a summary of the most common operations that are used in a RESTful API, together with their semantic meaning.\\

\begin{table}[h!]
	\centering
	\begin{tabularx}{\textwidth}{|X|X|X|X|X|}
		\hline
		\textbf{Resource type} & \textbf{GET} & \textbf{PUT} & \textbf{POST} & \textbf{DELETE} \\ \hline
		\emph{Collection} & Retrieve the collection. & Replace the collection. & Create a new entry in the collection. & Delete the collection.\\ \hline
		\emph{Item} & Retrieve the item. & Replace the item, create it if it does not exist yet. & \emph{Often not used.} & Delete the item.\\ \hline
	\end{tabularx}
	\caption{A summary of REST verbs and their usage when dealing with a resource collection or a single item.}
	\label{table:rest-api-operations}
\end{table}

\noindent Prior to implementation of this API, we can already define some of the resources and collections. In Tribler, we can identify torrents, channels, playlists and downloads as collections that should be available for retrieval or modification using the API. Other than that, we might define a debug collection that contains various statistics that are tracked by Tribler so developers can build debug tools, helpful during modification of features, performance measurements or when solving defects.\\\\
A RESTful API provides a flexible and high-level interface that allows developers to write applications that are using Tribler, ranging from a command-line interface (CLI) to appealing user interfaces. A great benefit is that the implementation of these utility applications is not bound to a specific programming language, providing implementation freedom for developers. Moreover, it allows to run the these applications and Tribler in separate environments, improving responsiveness, flexibility, performance and testability.

\subsection{Graphical User Interface}
The user interface is located at the highest level of the Tribler architecture stack. The user interface should be able to communicate with the Tribler core using the RESTful API as described in the previous section. A critical feature of the user interface is the ability to play and control a video. The current user interface uses the \emph{wxPython} compatibility library for the VLC player, however, these bindings are not functional on macOS due to differences in the platform architecture.\\\\
The implementation of this interface is not limited to one programming language, however, to be able to reuse prior-existing logic and in the Tribler code base, it is a decent choice to write the interface in Python. Since the programming language is high-level and relatively easy to learn, new developers can easily make modifications to the user interface. We might also consider to refactor the current user interface to support the API. This consideration will be analysed in more detail in Chapter \ref{chapter:towards-new-architecture}. 

\subsection{Requirements Conformance}
In Section \ref{sec:tribler-roadmap}, we defined three requirements that our architecture should meet: simplicity, flexibility and a focus on performance. After the proposal and discussion of the new architecture, we will now evaluate to what extent our proposed design meets the requirements we composed.

\subsubsection{\textbf{Simplicity}}
Our first requirement was simplicity. The architecture as depicted in Figure \ref{fig:tribler-core-architecture-55} is complex and has a steep learning curve for new developers. The proposed architecture is simpler by design and more divided into separate components, increasing re-usability and testability. When modifying a specific component in a layer, developers should not have to care about the layers below the layer that is being modified. In this sense, the architecture meets our requirement that it should be more comprehensible and usable for developers.

\subsubsection{\textbf{Flexibility}}
The APIs that are facilitating communication between the layers in the architecture, increases the flexibility. As described in Section \ref{subsec:communication-gui-libtribler}, the RESTful API allows a great amount of flexibility for developers. We strive towards an implementation where individual components can easily be toggled by developers using a configuration file. Some components should also be configurable by users, such as the credit mining mechanism.

\subsubsection{\textbf{Focus On Performance}}
One addition reason to split the architecture into different components, is based on performance engineering efforts performed on the system. By having components that are communicating with each other through an API, we are able to more easily extract and refactor these parts of the system in separate processes later on, thus increasing performance since different processes can utilize more CPU cores. Also, performance engineering is easier if the components can be modified in isolation.