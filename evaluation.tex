\chapter{Performance Evaluation of libtribler}
\label{chapter:experiments}

Technical debt has a negative impact on product quality in terms of defects and other structural quality issues\cite{tom2013exploration}: it is significantly harder to fix defects in a complex, unstructured system and more dangerous in a sense that one might introduce additional bugs when trying to solve one. Boosting performance is often achieved by minor or major refactoring efforts of system components to utilize another underlying model or structure. Modifications of the system is more involved when the system as whole is suffering from huge amounts of technical debt.\\\\
Now that we got rid of most of the technical debt identified in the Tribler core, the next step towards a stable, future-proof libtribler involves research efforts on the usability and performance of various components. We wish to quantify the performance of operation in libtribler to get an idea about the usability of the core in general. For various components in the core, we have no performance baseline to help us to make statements about usability. We will perform a number of experiments and for each experiment, we will present and discuss the observed results. The underlying reason for this experiment is twofold: on the one hand, we show that the performance of the system did not degrade to an  unacceptable extent due to our refactoring efforts. On the other hand, we use the performance measurements to create a baseline and to identify possible failures or issues that we will classify as future work.

\section{Environment Specifications}
\label{sec:environment-specifications}
The experiments performed in this Chapter are executed on a \emph{virtual private server (VPS)}. We wish to stay as close as possible to the specifications of a machine that an actual user could be using. A summary of the specifications of the machine used for most of the experiments described in this Chapter, is given in Table \ref{table:experiments-server-specifications}.

\begin{table}[h!]
	\centering
	\begin{tabular}{|l|l|}
		\hline
		\emph{CPU} & Intel Xeon CPU E5-2450 (2.50GHz, 4 cores)\\ \hline
		\emph{Memory} & 8GB 1000MHz \\ \hline
		\emph{Hard disk} & 50GB \\ \hline
		\emph{Operating system} & Ubuntu 15.10 \\ \hline
	\end{tabular}
	\caption{The specifications of the machine used for most of the experiments.}
	\label{table:experiments-server-specifications}
\end{table}

The experiments are not executed in an isolated, artificial environment but instead in the wild, using the deployed Dispersy network. While the obtained results may be different between users, this set up can be used to get more insights in the performance of Tribler from a user's perspective.\\\\
If not stated otherwise, the default Tribler configuration file values are used. These default values can be found in the \emph{defaults.py} file in the source code directory of Tribler\footnote{https://github.com/Tribler/tribler/blob/devel/Tribler/Core/defaults.py}. In this configuration file, all communities, except for the \emph{BarterCast} community, are initialized. Dispersy, the HTTP API and the video server are enabled during the experiments. All experiments are executed without running the \emph{wxPython} or Qt GUI.\\\\
Some of the experiments are built using a scenario file. In such a scenario file each line specifies a specific command of a peer at a specific point in time during the experiment. Our framework to run the experiments, Gumby, contains code to read scenario files, interprets the commands to be executed and to schedule these commands in Twisted. Several utility methods have been implemented to gather and write statistics to files in a processable and readable format that can be parsed by visualization tools such as \emph{R}\footnote{https://www.r-project.org}. Various Dispersy experiments are already using the scenario file framework, mainly in our \emph{AllChannel} experiment that runs on the DAS5 supercomputer. Before we executed the experiments in this Chapter, we first extended the usability of the scenario files to run and manage a Tribler session and we improved the framework with the addition of various commands to support the operations that are executed in the performed experiments in this Chapter. An overview of all implemented commands can be found in Appendix \ref{appendix:gumby-scenario-commands}. The flexibility of these scenario files gives developers a robust framework to use when conducting performance analysis, benchmarking and other kinds of scientific research with Tribler.

\section{Profiling Tribler on Low-end Devices}
\label{sec:profiling_tribler_lowend}
The implementation of a RESTful API gives developers a possibility to run and control Tribler from remote devices. For instance, one can run Tribler on a low-end, cheap devices such as a Raspberry Pi and use it to accumulate reputation in the Multichain by enabling the credit mining mechanism. Android is another example of a device that can run Tribler and during the last years, various research have been conducted to explore the possibilities of Tribler on Android devices\cite{sabee2014tribler}\cite{de2014android}. Executing and profiling Tribler on a low-end device with limited resources can yield much information about bottlenecks that might not be directly visible when running Tribler on a regular desktop or supercomputer.\\\\
The experiments described in this Section are all executed on a Raspberry Pi, third generation with 1GB LPDDR2 RAM, ARM Cortex-A53 CPU with 4 cores, a 1.2GHz CPU and 16GB storage on a microSD card. The installed operating system is Raspbian, an operating system specifically designed for the Raspberry Pi and derived from Debian, an operating system suitable for desktops.\\\\
Some preliminary exploration of the performance on the Raspberry Pi using the REST API has us suspected that the device is under heavy load when running Tribler. Monitoring the process for a while using the \emph{top} tool, reveals that the CPU usage is often around 100\%, completely filling up one CPU core. To get a detailed breakdown of execution time per method in the code base, the Yappi profiler has been used to gather statistics about the execution time of methods in Tribler. This profiler has been integrated in the \emph{twistd} plugin and can be started together with Tribler by passing an option. The output generated by the profiler is a \emph{callgrind} file that should be loaded and analysed by third party software. The breakdown of a 20-minute run is visible in Figure \ref{fig:yappi_breakdown}. This breakdown is generated using \emph{QCacheGrind}\footnote{https://sourceforge.net/projects/qcachegrindwin/}, a \emph{callgrind} file visualizer. We start this experiment with a clean state directory which is equivalent to the first boot of Tribler.

\begin{figure}[!h]
	\centering
	\includegraphics[width=1.0\columnwidth]{images/experiments/yappi_breakdown}
	\caption{The breakdown of a 20-minute run of Tribler on the Raspberry Pi.}
	\label{fig:yappi_breakdown}
\end{figure}

The file created by the Yappi profiler provides a detailed overview of the execution time of methods in Tribler and can be used as a tool to detect performance bottlenecks in the system. Referring to Figure \ref{fig:yappi_breakdown}, the column \emph{Incl.} denotes the inclusive cost of the function, in other words, the execution time of function itself and all the functions it calls. The column \emph{self} denotes only the execution time of the function itself, without considering callees. The other columns are self-explanatory and could be used to locate the respective function in the Tribler code base.\\\\
If we analyse the breakdown, we notice that Dispersy has a big impact on the performance of Tribler when running on the Raspberry Pi. The \emph{ecdsa\_verify} method (second method from the bottom) is dominating the runtime of Tribler: 45.81\% of the Tribler run time is spent inside this method. This specific method verifies the signature of an incoming Dispersy message and is invoked every time a signed message is received. Disabling cryptographic verification of incoming messages should improve the situation, however, this is a trade-off between security and performance: by not verifying incoming messages, fake messages by an adversary can be forged and are accepted in such a system.\\\\
To verify whether the system load when running Tribler decreases when we disable cryptographic verification of incoming messages, we measure the CPU usage of two different runs. Both runs start with a non-existing Tribler state directory and have a duration of ten minutes. In the first run, we are using the default configuration of Tribler, like in most of the other experiments described in this Chapter. In the second run, we disable verification of incoming messages in Dispersy. The CPU utilization over time of the two runs are displayed in Figure \ref{fig:raspi_cpu_usage}: on the horizontal axis, we show the time into the experiment and on the vertical axis, we display the CPU utilization in percentage. We emphasize that Tribler is limited to run on a single CPU core.\\\\

\begin{figure}[!h]
	\centering
	\includegraphics[width=1.0\columnwidth]{images/experiments/raspi_cpu_usage}
	\caption{The CPU utilization of one core on a Raspberry Pi device when running Tribler with and without cryptographic verification of incoming Dispersy messages.}
	\label{fig:raspi_cpu_usage}
\end{figure}

In Figure \ref{fig:raspi_cpu_usage}, some occurrences can be identified where the CPU usage appears to be slightly over 100\%. This is explained by the fact that some of the underlying code is designed to run on multiple processors. While the threading model of Tribler is limited to a single core, the Python interpreter might execute code on additional cores to improve performance. In the run where we enable cryptographic verification of incoming messages, the CPU usage is often 100\%, leading to a non-responsive system. When we disable message verification, we observe a somewhat lower CPU usage but overall, this utilization is still relatively high. Unfortunately, disabling incoming message verification is not enough to always guarantee a more usable and responsive system.\\\\
To detect other performance bottlenecks, we sort the report that has been generated by the Yappi profiler on the \emph{self} column to get insights in methods that are taking a long time to complete. This is visible in Figure \ref{fig:yappi_breakdown_self}. An interesting observation is that the Python built-in \emph{all} method takes up a significant amount of time (6.13\% of the runtime). The \emph{all} method takes an iterable object and returns \emph{true} if all objects of this collections resolve to a true value. Both the \emph{all} method and \emph{zip} method (also visible in Figure \ref{fig:yappi_breakdown_self}) is used in the \emph{\_resume\_delayed} method, indicating that this method might causing performance issues. Since further analysis of this method requires more knowledge of Dispersy, analysis and optimization of this method is considered future work and has been documented in GitHub issue 505\footnote{https://github.com/Tribler/dispersy/issues/505}.\\\\

\begin{figure}[!h]
	\centering
	\includegraphics[width=1.0\columnwidth]{images/experiments/yappi_breakdown_self}
	\caption{The breakdown of a 20-minute run of Tribler on the Raspberry Pi, sorted on the \emph{Self} column.}
	\label{fig:yappi_breakdown_self}
\end{figure}

To summarize, we demonstrated how adequate usage the Yappi profiler can lead to the detection of performance bottlenecks present in Tribler and Dispersy. Integration of the profiler in the twistd plugin makes it convenient for developers to run and analyse Tribler sessions under different circumstances and on a broad range of devices.

\section{Performance of the REST API}
The responsiveness of the REST API is directly influencing the user experience. If the response times of API calls is high, users of Tribler have to wait longer before their data is available and visible in the user interface. For this reason, we wish to make the API serve requests as fast as possible. The purpose of this section is to assess the performance of the API with a particular focus on latency of request response times. However, some other statistics will be considered such as average request time, standard deviation of the response times and observed bandwidth. These statistics will help us to get more insights in the performance of the REST API and the responsiveness of Tribler.\\\\
We make use of Apache JMeter\footnote{http://jmeter.apache.org} that is used to perform HTTP requests to Tribler and to gather and process performance statistics. The application allows to simulate a realistic user load, however, in this experiment we will limit the load to one user that performs a request to Tribler within a fixed time interval. The performed (GET) request will be targeted to a specific endpoint in the REST API: \emph{/channels/discovered}. This exact call happens when users are pressing the \emph{discover} menu button in the new Qt GUI and the response of the request consists of a JSON-encoded dictionary of all channels that Tribler has discovered so far. The returned response by this request can be rather large, especially if Tribler has been running for a long time and has discovered many channels (in our experiments, the average response size is around 613KB). When Tribler is processing the request, a database query is performed to fetch all channels that are stored in the persistent SQLite database.\\\\
We perform multiple experiments with different time intervals between requests made and a fixed total amount of 500 requests per experiment. First, we conduct the experiment with one request every second and we expect that the system should be able to hand this load and serve these requests in a timely matter. Next, the frequency of requests is increased to respectively 2, 5, 10 and 15 requests per second. These frequencies have been determined empirically and are based on the average request time, which appears to be several hundred milliseconds. Each experiment is started around five seconds after Tribler has started. During the experiment, we are using a pre-filled database with around 100.000 discovered torrents, 1.200 discovered channels and a subscription to 20 channels. A summary of the experimental results are visible in Table \ref{table:performance-api-results} where we present various request statistics.\\

\begin{table}[]
	\centering
	\begin{tabular}{|l|l|l|l|l|l|l|}
		\hline
		\textbf{Requests/sec.} & \textbf{Avg. (ms)} & \textbf{Std. dev. (ms)} & \textbf{Median (ms)} & \textbf{Min. (ms)} & \textbf{Max. (ms)} & \textbf{KB/S} \\ \hline
		\emph{1} & 241 & 476.34 & 76 & 56 & 4246 & 585.58\\ \hline
		\emph{2} & 170 & 327.86 & 68 & 58 & 3394 & 1127.04\\ \hline
		\emph{5} & 123 & 210.23 & 60 & 52 & 2082 & 2538.36\\ \hline
		\emph{10} & 115 & 238.72 & 60 & 50 & 2450 & 4120.70\\ \hline
		\emph{15} & 182 & 497.61 & 68 & 52 & 4937 & 3296.45\\ \hline
	\end{tabular}
	\caption{A summary of the experimental results when measuring the performance of the RESTful API.}
	\label{table:performance-api-results}
\end{table}

If we focus on the average request time (second column), the most interesting observation is that it appears that requests are served faster if we are performing requests at a faster rate, indicating that Tribler is able to handle the incoming requests well. This is surprising since one would expect this to be the other way around: when the frequency of requests is increased, the average request time is expected to increase since Tribler has more to process. The observed result is most likely explained by caching mechanisms data performed by the underlying database engine or Twisted.\\\\
The standard deviation of the request times in Table \ref{table:performance-api-results} (third column) is for all experiments rather high compared to the average request time. We suspect that this can be explained by the fact that Tribler is performing many different operations besides serving API requests. In particular, we think that Twisted is busy with processing other calls that have been scheduled earlier, causing the API calls to be processed later. To verify this, we ran the experiment again where we disable Dispersy, the component responsible for many calls in the reactor (as concluded in Section \ref{sec:profiling_tribler_lowend}). We perform five requests per second and 500 requests in total for this experiment. The observed results are illustrated in Figure \ref{fig:api-performance} where we display the time into the experiment on the horizontal axis in seconds and the request response time in milliseconds on the vertical axis.\\

\begin{figure}[h!]
	\centering
	\includegraphics[width=1.0\columnwidth]{images/experiments/request_times_comparison}
	\caption{The response times of API requests, in a Tribler session both with Dispersy enabled and disabled.}
	\label{fig:api-performance}
\end{figure}

In the left plot, the response times of the performed requests with a regular Tribler session is displayed (corresponding to the 5 requests/sec row in Table \ref{table:performance-api-results} ) whereas in the right plot, we display the response times of the run with a disabled Dispersy. Note the different scale on the vertical axis, indicating that the requests performed when Dispersy is disabled, are substantially faster. Indeed, the average request time of the right plot in Figure \ref{fig:api-performance} is 48 milliseconds, significantly lower than the average of the response times when Dispersy is enabled, namely 123 milliseconds. While both plots are showing a spiky pattern, the standard deviation of the right plot is 5.73 milliseconds and the standard deviation in the left plot is 210.23 milliseconds. We conclude that the variation in response times is much lower in the right plot and that DIspersy is producing a huge amount of work, introducing considerable amounts of latency when performing API requests.\\\\
We identified a key issue here: the latency of methods to be processed in Twisted is high, causing the processing of incoming requests to be delayed. This is not only a situation that occurs in the REST API: the same situation holds for Dispersy and the tunnel community where possibly many incoming connections have to be processed and served. A step in the right direction is to make sure that there are no big blocking calls scheduled in Twisted that are considerable amount of time to complete. When a method with a long execution time is executed, Tribler is unable to process other events during that period, leading to a less responsive system. Ongoing work is focussed on making the disk operations in Tribler non-blocking. This should reduce the latency of event processing and improve the responsiveness of the system in general.\\\\
Table \ref{table:performance-api-results} provides us with another interesting observation, namely that it appears that the bandwidth is reducing as the number of requests per second increases. This becomes more obvious if we plot the theoretical maximum bandwidth together with the observed bandwidth during the experiments, see Figure \ref{fig:api-bandwidth-performance}. In this Figure, we presented both the obtained bandwidth by running a regular Tribler session and one where Dispersy has been disabled. We assume that each request contains 613KB (627.712 bytes) of data in the response body. The theoretical maximum obtainable bandwidth is determined as $ b = 613 * n $ where $ n $ is the number of requests per second and $ b $ is the theoretical maximum bandwidth in KB/s. In practice, we will never reach this theoretical bandwidth since some time is required to initialize the HTTP connection in Tribler which we do not consider in our simple model. Figure \ref{fig:api-bandwidth-performance} clearly shows the impact of a running Dispersy on the bandwidth. Whereas we almost obtain the theoretical bandwidth when we disable Dispersy, the gap between the theoretical maximum and observed bandwidths becomes bigger in the run where we use a full session. When performing fifteen requests per second, the bandwidth even decreases, possibly due to the high system load.\\

\begin{figure}[h!]
	\centering
	\includegraphics[width=1.0\columnwidth]{images/experiments/api_bandwidth_performance}
	\caption{The theoretical maximum bandwidth compared to the observed bandwidth in the experiments (using a full Tribler session and disabled Dispersy).}
	\label{fig:api-bandwidth-performance}
\end{figure}

We conclude this experiment with the argumentation that we can use the API response times as a benchmarking tool to measure the responsiveness of the Tribler core. Using the Apache JMeter application, we can easily build a stress test and verify whether performance has increased or decreased after a specific set of modifications. Implementation of a performance regression framework is considered future work. 

\section{Start-up Experience}
\label{sec:startup-experience}
The first interaction with Tribler, is the process of booting the software. During this boot process, various operations are performed:
\begin{itemize}
	\item The Tribler state directory where runtime data are stored, is created and initialized with necessary files such as the SQLite database, the Dispersy member key pair and various configuration files.
	\item A connection to the persistent SQLite database is opened and initialized.
	\item Dispersy is initialized and the communities that are enabled in the configuration file are loaded.
	\item Various Tribler components are created, including the video streaming server, the RESTful API, the remote torrent handler, responsible for fetching torrent information from other peers and the \emph{leveldb} store, a key-value storage for torrent meta info.
\end{itemize}
The start-up process of the Tribler core proceeds sequentially and no parallel operations are implemented to speed up the process. Depending on the number of enabled components, the start-up time might vary.\\\\
To analyse the start-up time, we start Tribler 50 times in a row. The experiments are performed multiple times where in one experiment, the software is started for the first time, with no prior existing state directory. When starting Tribler with no prior existing state directory, a new one is created and the required files are initialized. In the other runs, a pre-filled database containing just over 100.000 torrents is used. This database is built by running Tribler idle for several hours, after subscribing to some popular channels to synchronize and discover as much torrents as possible. In both scenarios, there are no active downloads. A timer is started when the \emph{start} method of the \emph{Session} object is called and stopped when the notification that Tribler has started, is observed, allowing us to determine the total start-up time to a granularity of milliseconds. During the span of this thesis, there have been various changes to the start-up procedure of Tribler where code has been modified, removed and added. Since we would like to guarantee that our modifications do not significantly decrease the start-up speed and we make a comparison between the Tribler code in November '15 and July '16. The results are presented in Figure \ref{fig:startup_experiment}, where for each commit we compare, we present an empirical cumulative distribution function (ECDF) with the boot time in seconds on the horizontal axis and within each plot, the distribution of start-up times from a clean and pre-filled state directory.\\

\begin{figure}[!h]
	\centering
	\includegraphics[width=1.0\columnwidth]{images/experiments/startup}
	\caption{The start-up time of Tribler from a clean and pre-filled state using the code base in November '15 and July '16.}
	\label{fig:startup_experiment}
\end{figure}

The average start-up times for a clean and filled state using the November 2015 code base are 0.35 and 0.59 seconds respectively. For the July 2016 code, these values are 0.35 and 0.50 seconds. In both plots, It is clear that size of the Tribler database has impact on the time for Tribler to completely start. However, this impact is relatively minor since the system still starts within a second. We think that this statistic justifies removal of the splash screen that is shown in the old user interface: the relatively short time the splash screen would be visible in the new interface is so small that users would not even be able to read and interpret the content of the splash screen. In contrast to the old user interface, the new GUI starts Tribler and shows a loading screen after the interface has started. However, the difference is that users are able to already perform some actions before Tribler has started, such as the browsing of discovered content.\\\\
Whereas the boot times of the experiments performed with the November '15 code are very constant, we notice a larger variation in the runs with the code base from July '16, indicating that there is a component that has a high variation in initialization time during the start-up procedure. Further analysis learns us that this variation can be addressed to Dispersy, possibly caused by the initialization of one of the communities. However, further analysis of the boot time of Dispersy is outside the scope of this thesis work.

\section{Remote Content Search}
\label{sec:remote-content-search-experiment}
We wish to serve relevant information to users as fast as possible. To help users discover content they like, a remote keyword search has been implemented, allowing users to search for torrents and channels. Channel results are fetched by a query in the \emph{AllChannel} community whereas torrent results are retrieved by a query in the \emph{Search} community, however, for the experiments in this Section, we will focus on remote search for torrents since the amount of channels is rather small compared to the number of torrents available in the network.\\\\
Several experiments to verify the speed of a remote torrent search are discussed in this Section. A list of 100 search terms that have a high chance of triggering search results is used and each query is executed when there are at least twenty connected peers available in the \emph{Search} community (this condition is checked every second). The time-out period of the remote search is 60 seconds, indicating that incoming search results after this period are not regarded. This experiment is focussed on two performance statistics: the time interval until the first remote torrent search result comes in and the turnaround time of the search request, meaning the interval until the last search response arrives. We should note that users performing a remote search might see results earlier since a lookup query in the local database is performed in parallel (the performance of local search is discussed in Section \ref{sec:local-content-search}). The results of our experiment are visible in Figure \ref{fig:remote_search} where we created two ECDF plots with the distributions of time until the first response and time until the last response. On the horizontal axis, the measured time interval in seconds is visible.\\\\
Overall, the remote torrent search as implemented in Tribler is fast and performs reasonable. On average, there are 61 incoming search results for each performed query where the first torrent result takes on average 0.26 seconds to arrive. As we see in Figure \ref{fig:remote_search}, over 90\% of the first search results are available within a second. During our experiment, we always have the first incoming torrent result within 3.5 seconds. The plot displayed on the right shows the turnaround time of the request, indicating the time until the last response within our time-out period. On average, it takes 2.1 seconds for all torrent search results to arrive. In the plot, we see that in over 90\% of the search queries, we have all results within 10 seconds.\\

\begin{figure}[!h]
	\centering
	\includegraphics[width=1.0\columnwidth]{images/experiments/cdf_remote_search}
	\caption{The performance of remote content search, expressed in the time until the first response and time until last response.}
	\label{fig:remote_search}
\end{figure}

The same experiment has been performed in 2009 by Nitin et al. where 332 remote search queries have been performed. Their results are also presented in an ECDF in Figure \ref{fig:nitin_remote_search} where the time until the first response from any remote peer in the network is measured. The graph makes a comparison before and after a significant improvement to the input/output mechanism, causing messages to be exchanged at a faster rate. The observed average time until first response in 2009 is 0.81 seconds whereas the observed average time in our experiments is 0.26 seconds, more than three times as fast.

\begin{figure}[!h]
	\centering
	\includegraphics[width=0.7\columnwidth]{images/experiments/nitin_remote_search}
	\caption{The performance of remote content search, performed by Nitin et al. in 2009. The new remote search had an improved input/output mechanism, causing messages to be exchanged faster.}
	\label{fig:nitin_remote_search}
\end{figure}

\section{Local Content Search}
\label{sec:local-content-search}
In the previous Section, we demonstrated and elaborated the performance of the remote content search mechanism. Now, we will shift the focus to performance measurements of local content search, which is considered more trivial than the remote search counterpart due to the lack of network communication. In particular, our goal is to quantify the performance gain or loss when switching to the new relevance ranking algorithm that uses a newer search engine, as described in Chapter \ref{sec:relevance-ranking-algorithm}.\\\\
The set-up of this experiment is as follows: a database with just over 100.000 torrents is used. Around ten seconds after starting Tribler, we perform a local torrent search every second and we do this for 1.000 random keywords that are guaranteed to match at least one torrent in our database. We will measure both the time spent by the database lookup and the time it takes for the data to be post-processed after being retrieved from the database.. In the code used in November '15, this post-processing step involves determining the associated channels that are containing a specific torrent result. This experiment is performed for the old method that uses the \emph{Full Text Search 3 (FTS3)} engine and the new procedure that uses the more recent \emph{Full Text Search 4 (FTS4)} engine. According to the SQLite documentation, FTS3 and FTS4 are nearly identical, however, FTS4 contains an optimization where results are returned faster when performing searches with keywords that are common in the database. The results of the experiments with the old and new local search logic are visible in Figure \ref{fig:local-search-fts3-fts4}, presented in two ECDF plots with on the horizontal axis, the time of either the database query time (the red line) and the total time for the processing of results, including the query time (the blue/green line).\\

\begin{figure}[h!]
	\centering
	\includegraphics[width=1.0\columnwidth]{images/experiments/local_search_fts3_fts4}
	\caption{A comparison of the performance of local keyword searches between the old local search mechanism utilizing the FTS3 and FTS4 engine.}
	\label{fig:local-search-fts3-fts4}
\end{figure}

Local content search is very fast, delivering results in several milliseconds and low priority should be given to performance engineering on the local content search engine. We see that the two lines in the FTS3 and FTS4 plots have moved closer to each other which means that the total time of post-processing of torrent results has decreased. This is in line with our expectations since the new relevance ranking algorithm should be less computationally expensive than the old one. In addition, the new algorithm takes less factors in considering, for instance, the swarm health of the torrent. The increase in performance from FTS3 to FTS4 is visible but not very significant.\\\\
In 2009, Nitin et al. performed the same experiment where they used a database filled with 50.000 torrents. Their generated ECDF is displayed in Figure \ref{fig:local-search-nitin}. We notice that the performance of local search in our experiment is dramatically better than the performance obtained during the 2009 experiment. This can be explained by the fact that Tribler used a custom inverted index implementation when the experiment in 2009 was conducted. An inverted index is a data structure where a mapping is stored from words to their location in the database and is used on a large scale by search engines, including the FTS engine of SQLite. By utilizing this mapping when performing a full text search, we can get results in constant time. However, there is a slight overhead for maintaining and building the inverted index when new entries are added to the database, also impacting the size of the database disk file. The built-in FTS engine of SQLite is optimized to a large extent and clearly offers a higher performance than a custom implementation.

\begin{figure}[h!]
	\centering
	\includegraphics[width=1.0\columnwidth]{images/experiments/nitin_local_search}
	\caption{The performance of a local database query as verified by Nitin et al. in 2009.}
	\label{fig:local-search-nitin}
\end{figure}

\section{Video Streaming}
The embedded video player in Tribler allows users to watch a video that is being downloaded and the working is explained in more detail in Chapter \ref{subsubsec:video-server}. Video playback has been available since Tribler 4.0 and is implemented using the VLC library. One distinguishable feature is support for seeking so the user can jump to a specified time offset in the video. Video downloads have a special \emph{video on demand (VOD)} mode which means that the libtorrent piece picking mechanism uses a linear policy mode. In this mode, pieces are downloaded in a sequential order. When the user seeks to a position in the video, the prioritization of the pieces is modified, giving priority to pieces just after the specified seek position. Users also have the possibility to use an external video player that support playback of HTTP video streams.\\\\
The bytes are streamed to a VLC-compatible client using a HTTP stream. When Tribler starts, a video server is started if enabled in the configuration file. This server supports HTTP range requests which means that a specific part of a video file can be queried by using the HTTP \emph{range} header. This is useful when the user performs a seeking operation since only a specific part of the file has to be returned in the HTTP response, possibly saving a huge amount of bandwidth. If some requested pieces are not available, the video server will wait until these bytes are downloaded before returning these bytes in the response.\\\\
To improve user experience, we wish to minimize the delay that users experience when performing a seek operation in the video player. The experiment performed in this Section, will quantify this buffering delay. For this purpose, the well-seeded \emph{Big Buck Bunny}\footnote{https://peach.blender.org} movie will be downloaded. The movie file has a size of 885.6 MB and has a duration of 9 minutes and 56 seconds. We will perform various HTTP range requests using the \emph{curl} command line tool\footnote{https://curl.haxx.se}, immediately after starting the download in Tribler. For every run, we will request 10 megabyte of data and we will measure the total time it takes for each HTTP request to complete. The results are visible Table \ref{table:video_player_seek_times} where we specified the first requested byte and the time until the request has been fulfilled and the response is received.\\

\begin{table}[]
	\centering
	\begin{tabular}{|l|l|}
		\hline
		First byte               & Time until request done (sec) \\ \hline
		0                        & 11.6                  \\ \hline
		$ 1 * 10^9 $ & 64.4                  \\ \hline
		$ 2 * 10^9 $ & 64.6                  \\ \hline
		$ 3 * 10^9 $ & 65.9                   \\ \hline
		$ 4 * 10^9 $ & 100.6                   \\ \hline
		$ 5 * 10^9 $ & 115.6                   \\ \hline
		$ 6 * 10^9 $ & 115.8                  \\ \hline
		$ 7 * 10^9 $ & 12.2                  \\ \hline
		$ 8 * 10^9 $ & 66.6                   \\ \hline
		$ 9 * 10^9 $ & 52.4                   \\ \hline
	\end{tabular}
	\caption{Performance of the video server when requesting bytes at different offsets of the video being downloaded.}
	\label{table:video_player_seek_times}
\end{table}

Theoretically, we would expect around the same request time for each range request, assuming that the availability of each piece is high. When performing a seek operation in the video, the piece picking mechanism adjusts priorities and these prioritized pieces should start to download immediately. The experiments shows various anomalies in this mechanism where it might take up to two minutes for data to be available. Further investigation of this issue learns us that the video player always tries to download the first 10\% of the video file. We found out that this is intended behaviour of the code since VLC needs the information embedded in the file header first. This file header provides information about the file type, file duration and encoding used. There are some video formats where this kind of information is present at the end of the video file.\todo{conclusie??}\\\\

\section{Content Discovery}
\label{sec:content-discovery}
Content discovery is a key feature of Tribler. By running Tribler idle for a while, content is synchronized with other peers using the Dispersy messaging mechanism. When a user starts Tribler for the first time, there is no discovered content yet. We will verify the discovery speed of content after a first fresh start. The experiment is structured as follows: we measure the interval from the completion of start procedure to the moment in time where the first content is discovered. We perform these experiments for both torrents and channels and repeat this fifteen times. The results are visible in Figure \ref{fig:content_discovery_speed} where we created an ECDF with a distribution summary of the discovery times of torrents (green line) and channels (orange line).\\

\begin{figure}[!h]
	\centering
	\includegraphics[width=1.0\columnwidth]{images/experiments/content_discovery}
	\caption{The discovery time of the first channel and torrent after starting first starting Tribler.}
	\label{fig:content_discovery_speed}
\end{figure}

The delay of discovering the first channel is reasonably: this happens on average 18 seconds after start-up. In all runs, we have our first channel discovered within 35 seconds after Tribler starts. Discovery times of the first torrent is slightly slower and in all runs, the first torrent in a channel is discovered within 40 seconds. Figure \ref{fig:content_discovery_speed} suggests that a torrent discovery always happens after there is at least one discovered channel. This is true: after the channel is discovered, the \emph{PreviewChannel} community is joined where torrents are exchanged and discovered after a while.\\\\
In the old user interface, users were presented with a blank screen with no feedback about content that is being discovered in the background. In the new interface, the user is presented with a screen that informs the user that Tribler is discovering the first content. This screen is only shown the first time Tribler is started and is dismissed when there are five discovered channels after which the page with an overview of discovered channels is presented to the user.

\section{Channel Subscription}
When Tribler runs idle, not all available content in the network is discovered. The majority of content is discovered when users subscribe to a channel (in the old user interface, this is referenced to as marking a channel as favourite). When Tribler discovers a new channel, users are able to browse a preview of this channel. Internally, Tribler connects to the \emph{PreviewChannelCommunity} associated with that channel, a community derived from the \emph{ChannelCommunity}. In this preview community, the amount of torrents that are collected is limited. The \emph{ChannelCommunity} is joined the moment the user subscribes to a channel, after which the full range of content is synchronized. Removing the preview mechanism might significantly increases the resource usage of the Tribler session since the amount of incoming messages to be decoded and verified will increment.\\\\
The experiment as described in this Section, will focus on the discovery speed of additional content after the user subscribes to a specific channel and on the resource allocation when we are running Tribler without enabling a preview mechanism of channels. For the first experiment where we determine the discovery speed of additional content inside a channel, the twenty most popular channels (having the most subscribers) are determined. To get these channels, we have used a Tribler state directory with many discovered channels but void of any channel subscriptions. Exactly ten seconds after Tribler started, we subscribe to one of these popular channels and we measure the time interval between subscription to the channel and discovery of the first additional torrent in this channel. Tribler is restarted between every run and the state directory is cleaned so we guarantee a clean state of the system. The observed results are visible in Figure \ref{fig:channel-subscription}.

\begin{figure}[!h]
	\centering
	\includegraphics[width=1.0\columnwidth]{images/experiments/channel_subscription}
	\caption{An ECDF of discovery times of the first additional torrent after subscribing to a popular channel.}
	\label{fig:channel-subscription}
\end{figure}

The average discovery time of additional torrents after subscription to a channel is 36.8 seconds which is quite long, compared to the discovery speed of the first channel and torrent as described in Section \ref{sec:content-discovery}. The discovery times have a high variation as can be seen in Figure \ref{fig:channel-subscription}. This can be explained by the fact that immediately after subscribing to a channel, Tribler will connect to the \emph{ChannelCommunity} that is joined after subscription and peers have to be found.\\\\
To verify the impact of automatically subscribing to each channel when it is discovered, we perform a CPU usage measurement. In two idle runs of a Tribler session, both lasting for ten minutes, we measure the CPU usage every ten seconds using output provided by the \emph{top} tool. In the first run, a regular Tribler session is used where previews of discovered channels are enabled. In the second run, we bypass the preview of a discovered channel and immediately join the channel, synchronizing all available content. Both types of runs start with an empty state directory. The results of this experiment are visible in Figure \ref{fig:channel-subscription-cpu}.

\begin{figure}[!h]
	\centering
	\includegraphics[width=1.0\columnwidth]{images/experiments/subscribe_cpu_experiment}
	\caption{The CPU utilization of one core during a period of ten minutes with channel preview enabled and disabled.}
	\label{fig:channel-subscription-cpu}
\end{figure}

Whereas the CPU usage of the normal run is around 45\% on average, the CPU is quickly rising to 100\% utilization when we enable the auto-join mechanism of channels. This shows that it is infeasible to enable this auto-join feature if we still wish to guarantee a responsive system. One might limit the rate at which discovered torrents are fetched, however, this requires a feedback mechanism where we should notify other peers in the community to limit the amount of messages sent to the peer that is discovering content. Implementing of such as feature is outside the scope of this thesis work and is considered future work.

\section{Torrent Availability and Lookup Performance}
While specific information about torrents such as the name and file names are distributed within the Dispersy communities, this does not hold for the meta info about the torrent itself, which includes additional data such as available trackers and information about pieces. This meta info might be important for users since trackers provides information about the health of a torrent swarm. The experiments as explained in this Section, will investigate the torrent availability and lookup performance of meta info of torrents, either by using downloading them from remote peers in the Tribler network or from the \emph{Distributed Hash Table (DHT)}.

\subsection{Trivial File Transfer Protocol Handler}
When users are performing a remote torrent search, the first three incoming results are pre-fetched in the old user interface which means that the meta info of these torrents are fetched automatically. An incoming search result might contain information about remote peers (candidates) that have meta info of this torrent available. If candidates for a specific remote torrent result are present, an attempt to fetch the torrent meta info from this candidate is scheduled. This request is performed using the \emph{Trivial File Transfer Protocol (TFTP)}\cite{sollins1992tftp} which is a simplified version of the more sophisticated \emph{File Transfer Protocol (FTP)}\cite{postel1985rfc}, commonly used to transfer files over the internet. TFTP is also used to transfer meta data about torrent files such as thumbnails between peers, however, meta data of torrents is currently disabled in Tribler. The implementation of TFTP is located in the core package of the code base.\\\\
There has been no published studies yet about the performance of our TFTP implementation so we have no available reference material. The experiment performed in this Subsection will focus on the performance of TFTP when fetching meta info from remote peers. We start from a clean state directory and exactly one minute after starting Tribler, we perform a remote torrent search. For each incoming remote search result, we perform a TFTP request for each candidate attached to this result. We perform ten remote torrent search operations, with interval of 60 seconds between them. For every incoming result, we schedule a remote torrent lookup. After eleven minutes, we stop Tribler and gather the statistics of the TFTP sessions. The observed results are visible in Table \ref{table:tftp-performance}.\\

\begin{table}[h!]
	\centering
	\begin{tabular}{|l|l|}
		\hline
		\emph{Total requests scheduled} & 1008 \\ \hline
		\emph{Requests in queue} & 761 (75.5\%)\\ \hline
		\emph{Requests failed} & 106 (10.5\%)\\ \hline
		\emph{Requests succeeded} & 141 (14.0\%)\\ \hline
	\end{tabular}
	\caption{A breakdown of the performed requests during the TFTP performance measurement.}
	\label{table:tftp-performance}
\end{table}

We notice that the queue keeps growing: when our experiment is finished, 75.5\% of the initiated requests is still in the queue. The second observation is the high failure rate, when compared to the amount of succeeded requests (42.9\% if we do not consider the requests in the queue). We identified two underlying reasons for the failed requests. First, some of the requests timed out, possibly due to the fact that various remote peers are not connectible. A solution for this kind of failure would be a robust NAT puncturing method. The second reason is that the remote peer might not have the requested file in the local persistent storage. While this situation might seem unusual, it can happen if the remote peer has the requested torrent in the SQLite database but not in the meta info store, a separate persistent database. We can solve this by not returning this peer as candidate if the torrent is not available in the meta info store. This solution will also reduce the total bandwidth used by the TFTP component.\\\\
Next, we will focus on the turnaround time of successful TFTP requests, presented in the ECDF in Figure \ref{fig:tftp_performance-success}. Here, we notice the weird distribution of the turnaround times: we would expect that the total request times to fetch meta info using TFTP is somewhat constant, however, we see outgoing requests that take over 400 seconds to complete. This trend will probably continue if we did not stop the experiment after eleven minutes. The most reasonable explanation for this is that requests are added to the request queue at a faster rate than the processing speed of these requests. This also explains the values denoted in Table \ref{table:tftp-performance} where 75.5\% of the request are still in the queue after the experiment ends. Better support for parallel requests should help, however, this is considered further work.

\begin{figure}[!h]
	\centering
	\includegraphics[width=0.9\columnwidth]{images/experiments/tftp_performance}
	\caption{An ECDF of the performance of the torrent meta info download mechanism using TFTP in Tribler.}
	\label{fig:tftp-performance-success}
\end{figure}

\subsection{Distributed Hash Table}
\label{subsec:dht-experiment}
Another source of torrent meta info is the DHT. In the DHT we can lookup meta info of a torrent, identified by an infohash by invoking the \emph{download\_torrentfile} in the \emph{Session} object. In this Section, we will perform an experiment to get insights in the availability of torrent files and the performance of lookup operations in the DHT. This experiment is relevant to the user experience since users that want to determine whether specific content is interesting or not, they first might want to view meta info of the torrent file, including names and sizes of the files the torrent contains. This meta info should be available as soon as possible.\\\\
In the old wxPython user interface, the torrent file is fetched when the user single clicks on a torrent in a list of torrents, either when browsing through contents of a channel or after performing a remote keyword search. In addition, when executing a remote search, the first three top-results are pre-fetched since there is a higher probability that the user might be interested in them. For this experiment, a popular channel with over 5.000 torrents is considered and a subset of 1.000 torrent infohashes in this channel is determined. We start Tribler from a clean state and every 40 seconds, a DHT query is performed with one of the 1.000 random infohashes. The time-out period used in Tribler is 30 seconds, after which a failure callback is invoked and an error is displayed in the user interface to notify the user about the failed request. The results of this experiments are visible in the ECDF depicted in Figure \ref{fig:metainfo_fetch}.\\

\begin{figure}[!h]
	\centering
	\includegraphics[width=1.0\columnwidth]{images/experiments/metainfo_fetch}
	\caption{An ECDF of the lookup times of torrents in the DHT.}
	\label{fig:metainfo_fetch}
\end{figure}

We immediately notice that the failure rate of DHT lookups is quite high: 48.9\% of the lookup operations are timing out and never succeed. This issue might be addressed to dead torrents (when no peers in the DHT have this torrent information available) or private torrents (torrents which information is not exposed in the DHT). The amount of failures might be even higher in a less popular channel since the content in these channels are probable to be less seeded. As explained in the previous Subsection, the DHT is not the only source for torrents in Tribler and we might also fetch torrents from other peers using TFTP. Unfortunately, the approach of fetching meta info about torrents from other peers is only usable when searching for torrents. Caching and exchanging torrent candidates is not successful since the availability of candidates cannot be guaranteed.\\\\
The average lookup time of torrents that are successfully fetched from the DHT is 5.81 seconds which is reasonably fast. Additionally, Figure \ref{fig:metainfo_fetch} shows that a little over 90\% of the successfully fetched torrents are retrieved within 10 seconds.\\\\
To improve performance of meta info lookups, dead torrents should be handled correctly. One possible solution might be an implementation of a periodical check for each incoming torrent. By limiting the number of outstanding DHT requests, this approach does not create require much additional resources. To further improve performance, the result of DHT lookups might be disseminated to remote peers in the network. Torrents that are not successfully fetched from the DHT, could be hidden automatically in the user interface. The downside of this approach is that it might not give a realistic view of the availability of a torrent since their might be candidates which have a copy of this torrent available.