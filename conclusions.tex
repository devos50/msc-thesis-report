\chapter{Conclusions}
\label{chapter:conclusions}

% Tribler needs a culture change, this work tries to transform the way Tribler development is conducted...
Technical debt is a recurring problem in almost all large software engineering projects. The work in this thesis investigates the technical debt that has been accumulated by over 40 unique contributors over the last ten years of scientific research in the area of decentralized networks. After careful analysis, we concluded that the code base related to the user interface suffers from architectural impurity and an unmaintainable amount of technical debt of all kinds.\\\\
To better understand the root causes of the technical debt, we discussed the architectural evolution of Tribler over its lifespan, concluding that Tribler benefits from a new architecture that meets three requirements in place to reduce the amount of technical debt in the future: \emph{simplicity}, \emph{flexibility} and \emph{engineered for performance}. We started working towards this architecture by creating a new user interface, built using the mature and well-maintained Qt framework. Next, we created an extensive REST API that could be used to control Tribler by external developers. This provides migration possibilities, eventually leading to removal of the code base related to the old user interface.\\\\
While this solves the technical debt to a large extent, the core of Tribler is still suffering. During this thesis, we identified and solved different kinds of debt. Code debt has been identified with the SonarQube application and with some invasive refactoring efforts in the core, we reduced the amount of technical debt to just over one day. Testing debt has been handled by refactoring efforts in the testing framework mostly by splitting the tests into smaller parts while fixing various testing code smells. By updating and extending our CI environment with multi-platform unit tests and the improvement of Tribler installers, we reduced the amount of infrastructure debt. Architectural debt has been reduced by breaking dependencies in the Tribler core and by the implementation of the new user interface. Finally, we have set-up a new robust documentation structure, solving documentation debt.\\\\
Since technical debt has a negative impact on product quality in terms of defects and other structural quality issues, we identified where performance might have been lost due to the many architectural changes. We assessed the usability of various components found in libtribler and argued that our refactoring efforts did not degrade the performance of components\todo{uitbreiden}.\\\\
Prevention is the best medicine. We should learn from the mistakes made during the past years. Mandatory code reviews by other team members helps to improve one's code and to get a more critical attitude towards favouring short-term decisions over long-term agreements. We also propose that it is the responsibility of every developer to write code that is covered by the right amount of tests. By forcing a strict increasing code coverage policy, the code coverage metric is under control and can gradually be improved over time.\\\\
While this work is a step in the right direction to guarantee a robust platform for research in the area of decentralized networks, there is still much work to do. The performance of various components such as the video player and remote torrent handler can be improved to increase usability. The implementation of the trusted overlay located in the lower parts of the architecture in Figure \ref{fig:tribler7}.

%The main components as created in this thesis are a new user interface, built using the Qt framework, and a REST API, allowing Tribler to run on remote devices while giving developers high amounts of flexibility and ease when developing with Tribler. The testing environment has been improved with the addition of proper and stable unit tests and the tests are now executed on multiple platforms, allowing us to find defects due to platform incompatibilities earlier in the development process. Summarizing, this work transformed Tribler from an unattractive, unmaintained and untested system into a platform that is ready for development during the next decade of scientific research.\\\\
%While this is a step in the right direction, there still is a lot of work to do by the next generation of Tribler developers. \\\\
%Additional future work proposed is the implementation of the trust walker, which will become the foundations of the Tribler platform. This way, the Dispersy framework could be removed, allowing the deletion of much legacy code. Although not the main subject of this thesis, the amount of accumulated debt in Dispersy is also rather high. From a performance perspective, research should be conducted to see how the performance on low-end embedded devices could be improved. Since our main performance bottleneck is the limited amount of CPU power in a specific core, one of the proposed solutions is to increase utilization of multi-core architectures. This can be achieved by splitting the architecture of Tribler in separate components that can independently run on several cores, however, this requires one to think very carefully about the final design and communication structure.