\chapter{Conclusions}
\label{chapter:conclusions}

The work in this thesis investigates the technical debt that has been accumulated by over 40 unique contributors over the last ten years of scientific research in the area of decentralized networks. After a comprehensive discussion of the architectural evolution Tribler has made, A future-proof and robust architecture of Tribler is proposed, discussed and some parts of the new design have been implemented. The main components as created in this thesis are a new user interface, built using the Qt framework, and a REST API, allowing Tribler to run on remote devices while giving developers high amounts of flexibility and ease when developing with Tribler. The testing environment has been improved with the addition of proper and stable unit tests and the tests are now executed on multiple platforms, allowing us to find defects due to platform incompatibilities earlier in the development process. Summarizing, this work transformed Tribler from an unattractive, unmaintained and untested system into a platform that is ready for development during the next decade of scientific research.\\\\
While this is a step in the right direction, there still is a lot of work to do by the next generation of Tribler developers. We should learn from the mistakes made in the past years. Being critical towards the implementation of quick workarounds is one example of that. Mandatory code reviews by other team members helps to improve one's code and to get a more critical attitude towards favouring short-term decisions over long-term agreements. We also propose that it is the responsibility of every developer to write code that is covered by the right amount of tests. By forcing a strict increasing code coverage policy, the code coverage metric can be controlled and gradually improved over time.\\\\
Additional future work proposed is the implementation of the trust walker, which will become the foundations of the Tribler platform. This way, the Dispersy framework could be removed, allowing the deletion of much legacy code. Although not the main subject of this thesis, the amount of accumulated debt in Dispersy is also rather high. From a performance perspective, research should be conducted to see how the performance on low-end embedded devices could be improved. Since our main performance bottleneck is the limited amount of CPU power in a specific core, one of the proposed solutions is to increase utilization of multi-core architectures. This can be achieved by splitting the architecture of Tribler in separate components that can independently run on several cores, however, this requires one to think very carefully about the final design and communication structure.